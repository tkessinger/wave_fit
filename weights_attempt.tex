\documentclass[10pt]{revtex4}
%\documentclass[margin=3pt]{standalone}

\usepackage{tikz}
\usepackage{amsmath}
\usepackage{amssymb}
\usepackage{graphicx}
\usepackage{mathtools}

\newcommand{\shrug}[1][]{%
\begin{tikzpicture}[baseline,x=0.8\ht\strutbox,y=0.8\ht\strutbox,line width=0.125ex,#1]
\def\arm{(-2.5,0.95) to (-2,0.95) (-1.9,1) to (-1.5,0) (-1.35,0) to (-0.8,0)};
\draw \arm;
\draw[xscale=-1] \arm;
\def\headpart{(0.6,0) arc[start angle=-40, end angle=40,x radius=0.6,y radius=0.8]};
\draw \headpart;
\draw[xscale=-1] \headpart;
\def\eye{(-0.075,0.15) .. controls (0.02,0) .. (0.075,-0.15)};
\draw[shift={(-0.3,0.8)}] \eye;
\draw[shift={(0,0.85)}] \eye;
% draw mouth
\draw (-0.1,0.2) to [out=15,in=-100] (0.4,0.95); 
\end{tikzpicture}}


\begin{document}

We have some options for a birth-death process which it might be useful to consider.
Following Kendall (1948), let $\lambda$ be the birth rate and $\mu$ be the death rate.
The ``forward" equation for the probability $P_n(t)$ that there are $n$ individuals at time $t$ is
\begin{equation}
\partial_t P_n(t) = (n+1)\mu P_{n+1}(t) + (n-1)\lambda P_{n-1}(t) - n(\lambda+\mu)P_n(t).
\end{equation}
The generating function
\begin{equation}
\varphi(z,t) = \sum_{n = -\infty}^\infty P_n(t)z^n
\end{equation}
satisfies the PDE
\begin{equation}
\partial_t \varphi = (z-1)(\lambda z-\mu)\partial_z \varphi .
\end{equation}
The characteristic equation is given by
\begin{equation}
\frac{dz}{dt} = -\mu + (\lambda+\mu) z - \lambda z^2 .
\end{equation}
This is all old hat so far and can be neatly compared with, e.g., Desai and Fisher (2007) or the treatment in Allen (2004).
To study tunneling, one would be interested not in $P_n(t)$ but in the behavior of $P_{n,w}(t)$, the probability that there are $n$ individuals and a total weight $w$ at time $t$. To do this, one would have to introduce the new generating function 
\begin{equation}
\psi(z,y,t) = \sum_{n = 0}^\infty \sum_{w = 0}^\infty P_{n,w}(t) z^n y^w,
\end{equation}
which satisfies the PDE
\begin{equation}
\partial_t \psi = \left[ \lambda y z^2 - (\lambda + \mu)z + \mu \right] \partial_z \psi,
\end{equation}
with the boundary condition $\psi(z, y, 0) = zy$.
This is hard to solve, and it looks like nobody has done it in the general case.
Kendall (1948) tackles it only in the case of constant $\lambda$ and $\mu$ (it looks very similar to Dan's analysis, except that Dan let $w$ be continuous rather than discrete), and he briefly considers the ``Arley process", in which $\lambda$ is constant and $\mu$ is some constant multiple of $t$ (it seems this is a useful model for studying the behavior of cosmic rays).
Elsewhere the constant $\lambda$ and $\mu$ model is referred to as a ``Feller-Arley" process, which is nice, because it would just be too easy if people were vaguely consistent in their terminology. \shrug

Anyway, a few useful quantities generalize to the time dependent case, which Kendall (1948) obtains by using the cumulant generating function
\begin{equation}
K(u,v,t) = \log \psi(e^u, e^v, t)
\end{equation}
and matching terms in the resultant differential equations.
Let
\begin{equation}
\rho (t) = \int_0^t [\mu(\tau) - \lambda(\tau)] d\tau .
\end{equation}
The time dependent extinction probability is
\begin{equation}
P_0(t) = \frac{\int_0^t e^{\rho(\tau)} \mu(\tau) d\tau}{1 + \int_0^t e^{\rho(\tau)} \mu(\tau) d\tau}.
\end{equation}
(Sanity check: if the integral blows up, which it will if $\rho$ grows, the extinction probability is $1$; this is just a fancy way of saying $\mu(t) > \lambda(t)$ at long $t$.)
The distribution of extinction times $T$ will be
\begin{equation}
P(T) = \frac{e^{\rho(T)} \mu(T)}{\left(1 + \int_0^T e^{\rho(\tau)} \mu(\tau) d\tau\right)^2},
\end{equation}
which can be integrated to obtain the mean extinction time, and the mean cumulative population (``weight") $w$ at time $t$ will be
\begin{equation}
\mathbb{E}[w(t)] = 1 + \int^t_0 e^{-\rho (\tau)} \lambda(\tau) d\tau .
\end{equation}
We will probably want to send $t \to \infty$.

In principle these should be sufficient to give us an idea of how tunneling works in the case where the growth rate is set to $x - vt$.
Immediately we notice that, while the integral factor $\rho$ depends only on the difference between $\lambda$ and $\mu$, several other features of the weight depend on the exact form of $\lambda$ and $\mu$.
Past analyses such as Desai and Fisher (2007) have assigned $\mu$ the constant value of $1$, so that in effect time is ``scaled by" the death rate; Neher and Shraiman (2011) stuck the $-vt$ term in $\lambda$ and similarly set $\mu = 1$.
But we might well wish to stick the $-vt$ term in $\mu$, so that the death rate increases as time goes on (and to avoid the ``absurd" conclusion that the birth rate becomes negative).
We would have to be careful about what this implies for the ``time scale" of the process and how it compares to other population genetic models.
Whichever choice we make, we can immediately solve for
\begin{equation}
\rho(t) = \int_0^t (-x + v\tau) d\tau = - xt + vt^2/2.
\end{equation}
The astute reader will note that this leaves us to deal with an ugly Gaussian integral, which... sucks, but maybe we can get past it.
Suppose we choose $\lambda(t) = 1 + x$, $\mu(t) = 1 + vt$. Then
\begin{equation}
\mathbb{E}[T] = \int_0^\infty \left[ \frac{e^{\rho(T)} \mu(T) T}{\left(1 + \int_0^T e^{\rho(\tau)} \mu(\tau) d\tau\right)^2} \right]dT.
\end{equation}
We might be able to solve this. Let's do the integral in the denominator first:
\begin{equation}
\int_0^T e^{\rho(\tau)} \mu(\tau) d\tau = \int_0^T e^{- x\tau + v\tau^2/2} (1 + v\tau) d\tau = e^{-x} \left[\sqrt{\frac{\pi}{2v}} \mathrm{erfi} \left(\sqrt{\frac{vT^2}{2}}\right) + e^{vT^2/2} - 1\right].
\end{equation}
This looks pretty terrible, but we might be able to argue that it's bounded by something, or maybe it neatly cancels out with something.
The fact that $T$ is the bound of the integral in the denominator \emph{and} the variable of integration in the numerator means we can't simply send it to $\infty$ prematurely.
($\mathrm{erfi}(x) = -i \mathrm{erf}(ix)$ is the poorly named ``imaginary" error function.)
By inspection, I'm not even sure which expression in the result is dominant. Probably the ``inverse" Gaussian, but I'd have to check and see what $\mathrm{erfi}$ looks like.
We could certainly try to evaluate $\mathbb{E}[T]$ this way, but it seems like a fool's errand.

The mean weight (after sending $t \to \infty$) is
\begin{equation}
\mathbb{E}[w] = 1 + \int_0^\infty e^{x\tau-v\tau^2/2}(1+x)d\tau = 1 + \sqrt{\frac{\pi}{2v}}(1+x)e^{x^2/2v}\left[\mathrm{erf}(x/\sqrt{2v}) + 1\right].
\end{equation}
The second term (with the $\sqrt{\pi/2v}$ prefactor) looks like it's going to be dominant, and note that $\mathrm{erf}$ is bounded by $1$ anyway.
We might be able to tame some of these integrals by looking at asymptotics or doing a power series analysis.
Weirdly, choosing $\lambda(t) = 1 + x - vt$ instead causes the Wolfram integrator to freak out, which is odd, because the $1$ term is just an error function and the $x - vt$ term seems like it's just integration by parts. \shrug

Alternately, we know the ``mean" behavior of these bubbles for large $n$, which might be helpful ($n$ will, on average, be proportional to $e^{xt-vt^2/2}$, and $w$ will simply be the area under this curve, which unfortunately means we'll still have to deal with an error function, so maybe this isn't much more helpful after all).
This won't give us the bubble size distribution at all, but it will give us a sense of how long bubbles last, on average.
I'm actually not sure whether the distribution is really needed at all; could Dan have gotten the tunneling probability simply by considering the ``mean" behavior of transient bubbles?
(\ldots I guess not, since their ``mean" size will be decaying to zero using this argument, as their growth rate is always negative.)
In Desai and Fisher (2007), this trick of taking the mean behavior didn't work because lineages that establish are likely to have grown faster than an average lineage; but here, establishment is \emph{de facto} impossible (the process is ``transient", meaning extinction is assured).

In short, there might still be hope.
\end{document}